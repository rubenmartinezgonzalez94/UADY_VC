\documentclass{article}
\usepackage{listings} % Paquete para incluir código fuente
\usepackage{hyperref} % Paquete para manejar enlaces y referencias
\usepackage{amsmath} % Paquete para símbolos matemáticos
\usepackage{xcolor}
\usepackage{lstmisc}

\definecolor{codegreen}{rgb}{0,0.6,0}
\definecolor{codegray}{rgb}{0.5,0.5,0.5}
\definecolor{codepurple}{rgb}{0.58,0,0.82}
\definecolor{backcolour}{rgb}{0.95,0.95,0.92}

\lstdefinestyle{mystyle}{
    backgroundcolor=\color{backcolour},
    commentstyle=\color{codegreen},
    keywordstyle=\color{magenta},
    numberstyle=\tiny\color{codegray},
    stringstyle=\color{codepurple},
    basicstyle=\ttfamily\small,
    breakatwhitespace=false,
    breaklines=true,
    captionpos=b,
    keepspaces=true,
    numbers=left,
    numbersep=5pt,
    showspaces=false,
    showstringspaces=false,
    showtabs=false,
    tabsize=2
}

\lstset{style=mystyle}


\title{Descomposición en Valores Singulares para la Solución de Sistemas Sobredeterminados}
\author{RUBEN MARTINEZ GONZALEZ}
\date{\today}

\begin{document}

    \maketitle


    \section{Introducción}\label{sec:introduccion}
    La descomposición en valores singulares (SVD) es una técnica en álgebra lineal que se utiliza para resolver una variedad de problemas,
    incluyendo la resolución de sistemas de ecuaciones sobredeterminados.
    En este informe, se explica cómo se puede aplicar la SVD para resolver sistemas sobredeterminados de la forma $AX = 0$.


    \section{Desarrollo}\label{sec:desarrollo}
    Cuando se tiene un sistema de ecuaciones lineales de la forma AX=0, donde A es una matriz sobredeterminada
    (es decir, con más ecuaciones que incógnitas), la SVD puede proporcionar una solución que minimiza el error cuadrático.
    La descomposición en valores singulares de una matriz A es una factorización que puede escribirse como:
    \begin{equation}
        A = U \Sigma V^T\label{eq:equation}
    \end{equation}
    Donde:
    \begin{itemize}
        \item $U$ Es una matriz ortogonal de tamaño $m \times m$.
        \item $\Sigma$ Es una matriz diagonal de tamaño $m \times n$ (donde m es el número de filas de A y n es el número de columnas)
        con los valores singulares de A ordenados en forma decreciente en la diagonal.
        \item $V^T$ Es la matriz traspuesta de V, donde V es una matriz ortogonal de $n \times n$.
    \end{itemize}
    Para resolver el sistema de ecuaciones AX=0 donde A es sobredeterminada, podemos utilizar la descomposición en valores singulares
    para encontrar la solución que minimiza el error cuadrático.
    La solución se obtiene tomando el vector correspondiente a la columna de V asociada al menor valor singular de A.

    \noindent
    Este enfoque se basa en el hecho de que la SVD descompone la matriz A en tres componentes que representan la rotación,
    el cambio de escala y otra rotación.
    Así, la matriz V (o VT) captura la dirección en la cual la matriz A produce el menor cambio posible.


    \section{Implementación en Python}
    A continuación, se muestra una implementación simple en Python de la descomposición en valores singulares y la resolución de sistemas sobredeterminados utilizando la SVD.
    Nota: La implementación está desplegada en un notebook de Google Colab, el cual se puede acceder a través del siguiente enlace:
    \texttt{%
        \href{https://colab.research.google.com/drive/1CIy41HI3YFt20mPOnnGMbWKQPAHaVYAX?usp=sharing}{%
            Colab}%
    }



    \begin{lstlisting}[language=Python, caption=Implementación de SVD para resolver sistemas sobredeterminados]
import numpy as np

def svd(A):
    # Calcula A^T * A producto de la matriz transpuesta de A con A,
    ATA = np.dot(A.T, A)

    # Calcula los eigenvectors y eigenvalues de A^T * A
    eigenvalues, eigenvectors = np.linalg.eig(ATA)

    # Ordena los eigenvalues en función de los eigenvectors
    idx = eigenvalues.argsort()[::-1]
    eigenvalues = eigenvalues[idx]
    eigenvectors = eigenvectors[:,idx]

    # Calcula la matriz diagonal Sigma tomando la raíz cuadrada de los autovalores.
    sigma = np.sqrt(eigenvalues)

    # Calcula la matriz U dividiendo los autovectores por los elementos de Sigma
    #y luego multiplicando el resultado por la matriz original A
    U = np.dot(A, eigenvectors / sigma)

    # Calcula la matriz V
    V = eigenvectors

    return U, sigma, V.T

def solve_overdetermined_system(A):
    # Calcula la descomposición SVD de A
    U, sigma, VT = svd(A)

    # Encuentra el vector correspondiente a la columna de VT asociada al menor valor singular
    x = VT[-1, :]

    return x
    \end{lstlisting}

    \noindent
    El código anterior utiliza la biblioteca NumPy para calcular la descomposición en valores singulares de una matriz $A$ y encontrar la solución al sistema sobredeterminado.


    \section{Ejemplo de Uso}
    A continuación, se muestra un ejemplo de cómo se puede usar la implementación para resolver un sistema sobredeterminado:

    \begin{lstlisting}[language=Python, caption=Ejemplo de uso de la implementación]
# Creamos una matriz sobredeterminada A
A = np.array([[1, 2, 3],
              [4, 5, 6],
              [7, 8, 9],
              [10, 11, 12]])

# Resolvemos el sistema sobredeterminado AX = 0
solution = solve_overdetermined_system(A)

print("Solución del sistema sobredeterminado AX = 0:")
print(solution)

    \end{lstlisting}
    \begin{lstlisting}[language=Python, caption=Salida del ejemplo]
Solución del sistema sobredeterminado AX = 0:
[ 0.40824829 -0.81649658  0.40824829]
    \end{lstlisting}

    \noindent
    En este ejemplo, se crea una matriz sobredeterminada $A$ y se resuelve el sistema sobredeterminado $AX = 0$ utilizando la implementación de SVD.


    \section{Referencias}
    Para obtener más información sobre la descomposición en valores singulares y su aplicación en la resolución de sistemas sobredeterminados,
    se recomienda consultar las siguientes referencias:

    \begin{enumerate}
        \item  \texttt{%
            \href{https://https://ocw.mit.edu/courses/18-06-linear-algebra-spring-2010/resources/lecture-29-singular-value-decomposition/}{%
                Video Lecture by Gilbert Strang (MIT)}%
        }
        \item Gilbert Strang. \textit{Introduction to Linear Algebra}.
        \item Singular Value Decomposition (SVD).
          \texttt{%
            \href{https://www.cs.cmu.edu/~venkatg/teaching/CStheory-infoage/book-chapter-4.pdf}{%
                PDF link}%
        }
    \end{enumerate}

\end{document}
